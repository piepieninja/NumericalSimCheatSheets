\section{Generally Useful Maths}
\subsection*{Trig Properties}
$\sin^2x + \cos^2x = 1 \hspace{3mm} \sec x = \frac{1}{\cos x}$\\
$2 \sin x = \sin x \cos x$\\
$\tan x = \frac{\sin x}{\cos x} = \frac{1}{\cot x} \hspace{5mm} \csc x = \frac{1}{\sin x}$\\
$\frac{d}{dx}\sin x= \cos x \hspace{3mm} \frac{d}{dx}\cos x= - \sin x$\\
$\frac{d}{dx}\tan x = \sec^2 x \hspace{5mm} \frac{d}{dx}\cot x = - \csc^2 x$\\
$\frac{d}{dx} \arcsin x = \frac{1}{\sqrt{1 - x^2}}  \hspace{2mm}   \frac{d}{dx} \arccos x = \frac{-1}{\sqrt{1 - x^2}}  $\\
$\frac{d}{dx} \arctan x = \frac{1}{1 + x^2} \hspace{2mm} \frac{d}{dx} \sec x = \sec x \tan x $\\

\subsection*{Log \& Exp Properties}
$ \log x^n = n \log x \hspace{2mm} \log(\frac{1}{x}) = - \log x$\\
$\log_ax = \frac{\log_bx}{\log_ax} \hspace{5mm} \frac{d}{dx}e^{ax} = ae^{ax}$\\
$x^0 = 1 \hspace{3mm} x^n \cdot x^m = x^{n+m} \hspace{3mm} x^{-n} = \frac{1}{x^n}    $\\
$\log_a x^n = n \log_a x \hspace{5mm} \frac{e^{-n x}}{e^x} = e^{-(n+1)x}    $\\
$\log_a (\frac{x}{y}) = \log_a x - \log_a y \hspace{2mm}  $\\
$\log_a (xy) = \log_a x + \log_a y \hspace{2mm}  $\\
$\frac{d}{dx} \ln x = \frac{1}{x} \hspace{3mm} \frac{d}{dx} a^{g(x)} = \ln (a) a^{g(x)} g'(x)$\\
$\frac{d}{dx} a^{g(x)} = \ln (a) a^{g(x)} g'(x) \hspace{2mm} \frac{d}{dx} b^x = b^x \ln x  $\\
$\frac{d}{dx} e^{g(x)} = g'(x)e^{g(x)} \hspace{2mm} \frac{d}{dx}a^{x} = a^{x} \ln a$\\
$\frac{d}{dx} \log_a(g(x)) = \frac{g'(x)}{ln(a)g(x)}$\\

\subsection*{Other Derivative Rules}
$\frac{d}{dx} f(g(x)) = f'(g(x))g(x) \hspace{1mm}$\\
$\frac{d}{dx} f(x)/g(x) = \frac{(f'(x)g(x) - g'(x)f(x))}{g(x)^2}$\\

\subsection*{Useful Series}
$r^0 + r^1 + r^2 + r^3 = \frac{r^n - 1}{r - 1}$\\
for an alternating series the following will work to start:\\
$ \sum_{n=0}^{\infty} (-1)^n $ or $ \sum_{n=0}^{\infty} (-1)^{n+1} $\\

\subsection*{In Class Terminology}
the relative error formula:
$ \frac{|x - \hat{x}|}{x} $\\
more generally, with $\hat{x},\hat{y}$ being rounded terms we get relative error as:\\
$\frac{(x-y) - (\hat{x}-\hat{y})}{(x-y)} = relative \hspace{2mm} error$\\
these were represed strangely in class:\\
$ x' = f(t,x)  \hspace{3mm} x(2) = 1 \rightarrow t = 2, x=1 $\\
If $x'' =  xx'$ then $x''' = xx'' + x'x'$\\
When adding small number, it was mentioned in class that a $>=$ or $<=$ is
preferable to a $==$ when checking for values in a loop.
