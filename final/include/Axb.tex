\section{Linear Algebra Overview}

\subsection*{LU Decomposition Formulas}
Here $A = LU$, this would look like the following:
\[
A = LU, \hspace{5mm}
A =
\begin{bmatrix}
  a_{11} & a_{12} & a_{13} \\
  a_{21} & a_{22} & a_{23} \\
  a_{31} & a_{32} & a_{33} \\
\end{bmatrix}
\]
\[
L =
\begin{bmatrix}
  l_{11} & 0 & 0 \\
  l_{21} & l_{22} & 0 \\
  l_{31} & l_{32} & l_{33} \\
\end{bmatrix},
U =
\begin{bmatrix}
  u_{11} & u_{12} & u_{13} \\
  0 & u_{22} & u_{23} \\
  0 & 0 & u_{33} \\
\end{bmatrix}
\]
to make this fit on the cheatsheet I have broken $A$ into several vectors,
$A = [ a_1 | a_2 | a_3 ]$, so these parts of $A$ are:
\newline
$$
a_1 =
\begin{bmatrix}
l_{11} u_{11} \\
l_{21} u_{11} \\
l_{31} u_{11}
\end{bmatrix}
, \hspace{3mm}
a_2 =
\begin{bmatrix}
  l_{11} u_{12} \\
  l_{21} u_{12} + l_{22} u_{22} \\
  l_{31} u_{12} + l_{32} u_{22}
\end{bmatrix}
$$
$$
a_3 =
\begin{bmatrix}
  l_{11} u_{13} \\
  l_{21} u_{13} + l_{22} u_{23} \\
  l_{31} u_{13} + l_{32} u_{23} + l_{33} u_{33}
\end{bmatrix}
$$

% \[
% A =
% \begin{bmatrix}
%   l_{11} u_{11} & l_{11} u_{12} & l_{11} u_{13} \\
%   l_{21} u_{11} & l_{21} u_{12} + l_{22} u_{22} & l_{21} u_{13} + l_{22} u_{23} \\
%   l_{31} u_{11} & l_{31} u_{12} + l_{32} u_{22} & l_{31} u_{13} + l_{32} u_{23} + l_{33} u_{33}
% \end{bmatrix}
% \]

\subsection*{Ax = b, LU Decomposition Formulas}
Note that before $Ax = b$ and we also have $A = LU$, thus $LUx = b$. With a
substitution we get $Ux = y$, so $Ly = b$.

\subsection*{LU Decomposition Inverse Formula}
This is similar to normal $LU$ decomposition. The formula takes the form of
$AX = I$ where all matrices are n by n, so $X = [ x_1 | x_2 | ... | x_n ]$ with
the standard identity $I = [I_1 | I_2 | ... | I_n] $.

\subsection*{Gaussian Inverse Formula}
The gaussian inverse does the following: $[A | I] \rightarrow [I | A^{-1}]$

\subsection*{Cholesky Factorization Formulas}
This is similar to LU decomposition, only $U = L^T$. So: $A = L L^T$ which means
that:
\[
L =
\begin{bmatrix}
  l_{11} & 0 & 0 \\
  l_{21} & l_{22} & 0 \\
  l_{31} & l_{32} & l_{33} \\
\end{bmatrix},
L^T =
\begin{bmatrix}
  l_{11} & l_{21} & l_{31} \\
  0 & l_{22} & l_{32} \\
  0 & 0 & l_{33} \\
\end{bmatrix},
\]
\vspace{42mm}
\[
A =
\begin{bmatrix}
  l_{11}^2 & l_{11}l_{21} & l_{11}l_{31} \\
  l_{21}l_{11} & l_{21}^2 + l_{22}^2 & l_{21}l_{31} + l_{22}l_{32}\\
  l_{31}l_{11} & l_{31}l_{21} + l_{32}l_{22} & l_{31}^2 + l_{32}^2 + l_{33}^2
\end{bmatrix}
\]
Then solving for the system and getting a cleaner $L$ at the end gives the formula:
\[
L =
\begin{bmatrix}
\sqrt{a_{11}} & 0 & 0 \\
\frac{a_{21}}{l_{11}} & \sqrt{a_{22} - l_{21}^2} & 0 \\
\frac{a_{31}}{l_{11}} & \frac{a_{32} - l_{31}l_{21}}{l_{22}} & \sqrt{A_{33} - l_{31}^2 - l_{32}^2}
\end{bmatrix}
\]

But the actual Cholesky factorization is just the matrix $L$.

\subsection*{Gaussian Elimination Formulas}
Gassian elimination is an $Ax = b$ solving method. It makes an upper triangual matrix
with the form: $[A | b] $
\newline
% \vspace{130mm}
\\~














% yee
