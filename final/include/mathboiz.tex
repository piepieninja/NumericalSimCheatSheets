\section{Taylor, Maclaurin, \& Euler}

\subsection*{Taylor Series}
The Taylor series is a sum of derivatives of increasing order that equate to a function. The formula for
the Taylor series of $f(x)$ evaluated at $a$ is:\\
$f(x) = \sum_{n=0}^{\infty} \frac{f^{(n)}(a)}{n!}(x - a)^n  = f(a) + f'(a)(x-a) + \frac{f''(a)}{2!} (x - a)^2 + \frac{f'''(a)}{3!} (x-a)^3 $\\

\subsection*{Alternating Series Theorem}
If $a_x \geq a_2 \geq ... \geq a_n \geq 0$ for all $n$ and $lim_{n \rightarrow \infty} a_n = 0$
then the alternating series $a_1 - a_2 + a_3 - a_4 + ...$ so,\\
$S = lim_{n \rightarrow \infty} S_n = \sum_{k=1}^{\infty} (-1)^{k-1} a_k = lim_{n \rightarrow \infty} \sum_{k=1}^{\infty} (-1)^{k-1} a_k$\\
Here $S$ is the sum and $S_n$ is a partial sum. we note that, for all $n$, $|S - S_n| \leq a_{n+1}$

\subsection*{Taylor's Method for ODEs}
This method takes advantage of the previously mentioned series. here this is some
step size $h$ that we take from some $f(x)$ value. This is the Initial Value Problem (IVP).\\
$f(x + h) = f(x) + f'(x)h + \frac{1}{2!}f''(x)h^2 + \frac{1}{3!}f'''(x)h^3 + \hspace{2mm} ...$\\

\subsection*{Maclaurin Series}
The Maclaurin series is just the Taylor series at the special case where $x=0$.
This gives the following:\\
$f(x) = f(0) + f'(0) + \frac{x^2}{2!}f'''(0) + \frac{x^3}{3!}f'''(0) + \frac{x^4}{4!}f''''(0) + \hspace{2mm} ... = \sum_{n=0}^{\infty} \frac{f^{(n)}(0)}{n!}x^n$\\

\subsection*{Euler's Method for ODEs}
This method is just a Taylor series of order $1$ with the same step term $h$,
though many steps can be taken:\\
$ f(x + h) = f(x) + f'(x)h $\\

\subsection*{Modified Euler's Method}
This is the equation:\\
$x_{n+1} = x_n +h \frac{f(t_n,x_n) + f(t_{n+1},x_{n+1})}{2}$\\
$= x_n = \frac{h}{2}[ x'_{n} + x'_{n+1} ]$\\

\subsection*{Error Terms}
We note that Taylor's theorem in terms of $x+h$ is:\\
$ f(x+h) = \sum_{k=0}^{n} \frac{f^{k}(x)}{k!}h^k + E_{n+1} $\\
Thus, error terms are of the form:\\
$ E_{n+1} = \frac{f^{n+1}(\xi)}{(n+1)!}h^{n+1}$\\

It pays off to look at the term more specifically for the problem. A lot of times the
error term takes the form $\frac{(n+1)^2}{(n+1)!}$ or $\frac{(n+1)^2}{(n+1)}$.

It is important to note that we only care about the $0.5 \cdot 10^n$ if our desired
accuracy is to the $n$th decimal. Thus we set $E_{n+1} < 0.5 \cdot 10^n$ the $n+1$
portion of this is \textbf{very important!} To reiterate:

$ E_{n+1} = \frac{f^{n+1}(\xi)}{(n+1)!}h^{n+1} $ or $E_{n+1} = \frac{x^{n+1}}{(n+1)!}$\\

\subsection*{First Derivative Formulas}
For Taylor's Theorem, the forward difference formula is:\\
$f'(x) = \frac{f(x+h) - f(x)}{h} + $ error of $O(h)$\\
For Taylor's Theorem, the backwards difference formula is:\\
$f'(x) = \frac{f(x) - f(x - h)}{h} + $ error of $O(h)$\\
For Taylor's Theorem, the central difference formula is:\\
$f'(x) = \frac{f(x + h) - f(x - h)}{2h} + $ error of $O(h^2)$\\









% yee
