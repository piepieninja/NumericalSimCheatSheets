\section{Loss of Significance}


\subsection*{Loss of Precision Theorem}
The general form of the theorem is as follows:\\
$x$ and $y$ are floating point numbers such that $x > y > 0$,
the theorem states that given:
$ 2^{-p} \leq 1 - \frac{y}{x} \leq 2^{-q} $\\
there are at most $p$ and at least $q$ digits lost in the subtraction $x-y$.\\

practically speaking, we view the equation as: $E(x) = f(x) - g(x)$. If
we notice this approaches $0$ we have a concern of loss of precision at that point.
to find that point we typically view the max loss acceptable as $1$, so we set
the euqation to $\frac{g(x)}{f(x)} = \frac{1}{2}$. We find the $x = z$ values that cause
the $\frac{1}{2}$ flip and use a Taylor method there and use the normal formula elsewhere.
We're just avoiding the loss of precision as $x \rightarrow z$.

\subsection*{Rationalizing Numerators}
In some cases we want to rationalize a numerator to avoid a loss of significance.
The general form form for radicals in a demonitor is:\\
$\sqrt[\leftroot{-2}\uproot{2}k]{x^n + r} + c  \cdot \frac{\sqrt[\leftroot{-2}\uproot{2}k]{x^n + r} - c}{\sqrt[\leftroot{-2}\uproot{2}k]{x^n + r} - c}   =  \frac{x^n + r -2c}{\sqrt[\leftroot{-2}\uproot{2}k]{x^n + r} - c}$\\

\subsection*{Small Numbers}
If a set of small numbers $\{s_0, s_1, ... , s_i\}$ is each on the order of $10^{n+1}$ decimal places but
a large large number $l$ is on the order of $10^{n}$ decimal places, it is better
to add $\sum_{k=0}^{i} s_k$ small numbers there are before
adding an $l$ large number.
