\section{Base Conversion}
\subsection*{Decimal to Binary}
For this simply find the place of the largest binary number that (of the form $2^n$)
that is within the number. Successivley subtract these numbers while keeping track of their
place to generate the binary number.

\subsection*{Binary to Decimal}
For this notice that each place in the decimal number has a
corresponding power of 2. If the decimal number has a floating
point then the power is negative counting from zero. This generates a
sum of the form:\\
$ 2^{n} + \hspace{1mm} ... \hspace{1mm} + 2^2 + 2^1 + 2^{-1} + 2^{-2} + \hspace{1mm} ... \hspace{1mm} + 2^{-m} $\\
Where $n$ is the most significant digit and $m$ is the least. The $2^{-1}$ term
is the beginning of the floating point numbers.

\subsection*{Binary to Octal}
Simply follow the table:
$ 000 \rightarrow 0 \hspace{2mm} 001 \rightarrow 1 \hspace{2mm} 002 \rightarrow 2 \hspace{2mm} 003 \rightarrow 3 $
$ \hspace{2mm} 004 \rightarrow 4 \hspace{2mm} 005 \rightarrow 5 \hspace{2mm} 006 \rightarrow 6 \hspace{2mm} 007 \rightarrow 7 $\\

\subsection*{Binary to Hex}
This identical to the Octal method, the Hex symbols range from $0$ to $F$ and binary
from $0000$ to $1111$. Simply count up un binary and there is a simple conversion.

\subsection*{One \& Two's Complement}
% https://www.cs.cornell.edu/~tomf/notes/cps104/twoscomp.html
% https://www.geeksforgeeks.org/1s-2s-complement-binary-number/
The one's complement of a bitstring is, simply, the inverse of that bitstring.
i.e. all $1$\'s become $0$\'s and vice versa. The two's complement of a bitstring
is the one's complement $+1$ at the end, so that (sometimes) there is a cascade
of digit flips that occur. 
