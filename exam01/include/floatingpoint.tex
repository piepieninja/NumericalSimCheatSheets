\section{IEEE Floating Points}
\subsection*{Definitions}
$s = $ signed bit, $c =$ based exponent, $F =$ fraction. The general form
for this is $(-1)^s \cdot 2^{c - 127} \cdot 1.F$, for both $|s| = 1$\\
For single precision: $|c| = 8$, $|F| = 23$\\
For double precision: $|c| = 11$, $|F| = 52$\\

\subsection*{Converting to IEEE Format}
A number will have the form $D_n ... \hspace{1mm} D_1 D_0 . F_0 F_1 \hspace{1mm} ... F_m$,
to start we need to shift the values left (normalize) so that the number is now
of the form: $D_n . F_0 F_1 \hspace{1mm} ... F_{m + (n-1)} \cdot 10^{n-1}$.

\subsection*{Example}
Converting the number $-42.125$ to binary floating point with single precision:\\
$0.125 \cdot 2 = [0].25 \rightarrow 0.25 \cdot 2 = [0].5 \rightarrow 0.5 \cdot 2 = [1].0$
Thus the fractional part is: $0.001$, the nonfractional part is:\\
$\frac{1}{32} \hspace{1mm} \frac{0}{16} \hspace{1mm} \frac{1}{8} \hspace{1mm} \frac{0}{4} \hspace{1mm} \frac{1}{2} \hspace{1mm} \frac{0}{1} \rightarrow 101010$\\
The full value, $101010.001$, when normaized is: $1.01010001 \cdot 10^5$. We find the $c$
term with $2^{c - 127} = 2^5 \rightarrow c = 132$ and we already have $F = .01010001$
This gives us the following number with single precision:\\
$[1][10000100][01010001 ... b_{23}]$\\
